\documentclass{article}
% generated by Madoko, version 1.0.0-rc7
%mdk-data-line={1}


\usepackage[heading-base={2},section-num={False},bib-label={True}]{madoko2}


\begin{document}



%mdk-data-line={5}
\mdxtitleblockstart{}
%mdk-data-line={5}
\mdxtitle{\mdline{5}图像处理第四次作业}%mdk
\mdxauthorstart{}
%mdk-data-line={10}
\mdxauthorname{\mdline{10}SeanChense}%mdk
\mdxauthorend\mdtitleauthorrunning{}{}\mdxtitleblockend%mdk

%mdk-data-line={7}
\section{\mdline{7}\mdinline{width=}{\href{images/1.png}{1}}.\hspace*{0.5em}\mdline{7}第一题、第二题}\label{section}%mdk%mdk

%mdk-data-line={9}
\noindent\mdline{9}\includegraphics[keepaspectratio=true,width=\dimmin{}{\dimwidth{0.90}}]{images/1}{}\mdline{9}
\mdline{10}\includegraphics[keepaspectratio=true,width=\dimmin{}{\dimwidth{0.90}}]{images/2}{}\mdline{10}
\mdline{11}\includegraphics[keepaspectratio=true,width=\dimmin{}{\dimwidth{0.90}}]{images/3}{}\mdline{11}
\mdline{12}\includegraphics[keepaspectratio=true,width=\dimmin{}{\dimwidth{0.90}}]{images/4}{}\mdline{12}
\mdline{13}\includegraphics[keepaspectratio=true,width=\dimmin{}{\dimwidth{0.90}}]{images/5}{}\mdline{13}%mdk

%mdk-data-line={22}
\begin{quote}%mdk

%mdk-data-line={23}
\begin{mdcenter}%mdk

%mdk-data-line={24}
\noindent\mdline{24}\textbf{Main Script}%mdk
%mdk
\end{mdcenter}%mdk
%mdk
\end{quote}%mdk

%mdk-data-line={28}
\begin{mdblock}{width=\dimwidth{0.50}}%mdk
\mdhr{}%mdk
\end{mdblock}%mdk
\begin{mdpre}%mdk
\noindent periodPic~=~zeros(250,~250);\\
for~i=-5:5\\
\preindent{4}periodPic(20:230,~125~+~17*i:125~+~17*i~+~6)~=~255;\\
end\\
\\
threeFilter~=~[1,~1,~1;\\
\preindent{15}1,~1,~1;\\
\preindent{15}1,~1,~1]/9.0;\\
\preindent{11}\\
fiveFilter~~=~[1,~1,~1,~1,~1;\\
\preindent{15}1,~1,~1,~1,~1;\\
\preindent{15}1,~1,~1,~1,~1;\\
\preindent{15}1,~1,~1,~1,~1;\\
\preindent{15}1,~1,~1,~1,~1]/25.0;\\
\preindent{11}\\
nineFilter~~=~[1,~1,~1,~1,~1,~1,~1,~1,~1;\\
\preindent{15}1,~1,~1,~1,~1,~1,~1,~1,~1;\\
\preindent{15}1,~1,~1,~1,~1,~1,~1,~1,~1;\\
\preindent{15}1,~1,~1,~1,~1,~1,~1,~1,~1;\\
\preindent{15}1,~1,~1,~1,~1,~1,~1,~1,~1;\\
\preindent{15}1,~1,~1,~1,~1,~1,~1,~1,~1;\\
\preindent{15}1,~1,~1,~1,~1,~1,~1,~1,~1;\\
\preindent{15}1,~1,~1,~1,~1,~1,~1,~1,~1;\\
\preindent{15}1,~1,~1,~1,~1,~1,~1,~1,~1]/81.0;\\
\preindent{11}\\
\%~periodPic~=~imread('cameraman.tif');\\
subplot(2,3,2)\\
imshow(periodPic);\\
title('Original~Image');\\
subplot(2,3,4)\\
imshow(SCFilter(periodPic,~threeFilter));\\
title('3X3~均值滤波');\\
subplot(2,3,5)\\
imshow(SCFilter(periodPic,~fiveFilter));\\
title('5X5~均值滤波');\\
subplot(2,3,6);\\
imshow(SCFilter(periodPic,~nineFilter));\\
title('9X9~均值滤波');figure;\\
\\
\\
subplot(2,3,2)\\
imshow(periodPic);\\
title('Original~Image');\\
subplot(2,3,4)\\
imshow(SCOrderFilter(periodPic,~1,~3));title('3x3~最大滤波');\\
subplot(2,3,5)\\
imshow(SCOrderFilter(periodPic,~1,~5));title('5x5~最大滤波');\\
subplot(2,3,6);\\
imshow(SCOrderFilter(periodPic,~1,~9));title('9x9~最大滤波');figure;\\
\\
subplot(2,3,2)\\
imshow(periodPic);\\
title('Original~Image');\\
subplot(2,3,4)\\
imshow(SCOrderFilter(periodPic,~0,~3));title('3x3~中点滤波');\\
subplot(2,3,5)\\
imshow(SCOrderFilter(periodPic,~0,~5));title('5x5~中点滤波');\\
subplot(2,3,6);\\
imshow(SCOrderFilter(periodPic,~0,~9));title('9x9~中点滤波');figure;\\
\\
subplot(2,3,2)\\
imshow(periodPic);\\
title('Original~Image');\\
subplot(2,3,4)\\
imshow(SCHarmonicMeanFilter(periodPic,~0,~int32(3)));title('3x3~谐均值滤波');\\
subplot(2,3,5)\\
imshow(SCHarmonicMeanFilter(periodPic,~0,~int32(5)));title('5x5~谐均值滤波');\\
subplot(2,3,6);\\
imshow(SCHarmonicMeanFilter(periodPic,~0,~int32(9)));title('9x9~谐均值滤波');\\
figure;\\
\\
subplot(2,3,2)\\
imshow(periodPic);\\
title('Original~Image');\\
subplot(2,3,4)\\
imshow(SCHarmonicMeanFilter(periodPic,~1,~int32(3)));title('3x3~逆谐均值滤波');\\
subplot(2,3,5)\\
imshow(SCHarmonicMeanFilter(periodPic,~1,~int32(5)));title('5x5~逆谐均值滤波');\\
subplot(2,3,6);\\
imshow(SCHarmonicMeanFilter(periodPic,~1,~int32(9)));title('9x9~逆谐均值滤波');%mdk
\end{mdpre}\noindent\mdline{113} \mdline{113} 

%mdk-data-line={113}
\begin{quote}%mdk

%mdk-data-line={114}
\begin{mdcenter}%mdk

%mdk-data-line={115}
\noindent\mdline{115}\textbf{SCHarmonicMeanFilter.m}%mdk
%mdk
\end{mdcenter}%mdk
%mdk
\end{quote}%mdk

%mdk-data-line={117}
\begin{mdblock}{width=\dimwidth{0.50}}%mdk
\mdhr{}%mdk
\end{mdblock}%mdk
\begin{mdpre}%mdk
\noindent\%~SeanChense\\
\%~style:~0~表示谐均值滤波\\
\%~~~~~~~~1~表示逆谐均值滤波\\
function~[~result~]~=~SCHarmonicMeanFilter(f,~style,~dim)\\
int\_dim~=~int32(dim);\\
[m,~n]=~size(f);\\
result~=~zeros(m,~n);\\
if~style~==~0\\
\preindent{4}for~i~=~1+int\_dim/2:m-int\_dim/2\\
\preindent{8}for~j~=~1+int\_dim:n-int\_dim/2~~~~~~~~\\
\preindent{12}con=0;~s1=0;\\
\preindent{12}for~k1~=~i-int\_dim/2:i+int\_dim/2\\
\preindent{16}for~p1~=~j-int\_dim/2:j+int\_dim/2~~~~~~~~~~~~~~~\\
\preindent{24}con~=~con+1;\\
\preindent{24}if~f(k1,p1)==0\\
\preindent{28}s1~=~s1+0;\\
\preindent{24}else\\
\preindent{28}s1=s1+(1/f(k1,p1));\\
\preindent{24}end~~~~~~~~~~~~~~~~\\
\preindent{16}end\\
\preindent{12}end~~~~~~~~\\
\preindent{12}temp~=~con/s1;\\
\preindent{12}if~temp~\textgreater{}~255\\
\preindent{16}temp~=~0;\\
\preindent{12}end\\
\\
\preindent{12}result(i,j)=temp;\\
\preindent{8}end~~~~\\
\preindent{4}end\\
\\
else~\\
\preindent{4}Q=-1.5;\\
\preindent{4}for~i~=~1+int\_dim/2:m-int\_dim/2\\
\preindent{8}for~j~=~1+int\_dim/2:n-int\_dim/2~~~~~~~~\\
\preindent{12}con=0;~s1=0;~s2=0;\\
\preindent{12}for~k1~=~i-int\_dim/2:i+int\_dim/2\\
\preindent{16}for~p1~=~j-int\_dim/2:j+int\_dim/2\\
\preindent{24}con~=~con+1;\\
\preindent{24}if~f(k1,~p1)~==~0\\
\preindent{28}s1=s1+0;\\
\preindent{28}s2=s2+(f(k1,p1)\textasciicircum{}(Q+1));\\
\preindent{24}else\\
\preindent{28}s1=s1+(f(k1,p1)\textasciicircum{}Q);\\
\preindent{28}s2=s2+(f(k1,p1)\textasciicircum{}(Q+1));~\\
\preindent{24}end\\
\preindent{16}end\\
\preindent{12}end\\
\preindent{12}\\
\preindent{12}\\
\preindent{12}if~s1~\textasciitilde{}=~0~\&\&~s2~\textasciitilde{}=~0\\
\preindent{16}temp~=~s2*1.0/(s1);\\
\preindent{16}if~temp~\textgreater{}~255\\
\preindent{20}result(i,j)=~255;\\
\preindent{16}else~\\
\preindent{20}result(i,j)=temp;\\
\preindent{16}end\\
\preindent{16}\\
\preindent{12}else\\
\\
\preindent{12}end\\
\preindent{8}end~~~~\\
\preindent{4}end\\
end\%~if~~~~\\
end%mdk
\end{mdpre}
%mdk-data-line={186}
\section{\mdline{186}\mdinline{width=}{\href{images/2.png}{2}}.\hspace*{0.5em}\mdline{186}第三题}\label{section}%mdk%mdk

%mdk-data-line={189}
\noindent\mdline{189}\includegraphics[keepaspectratio=true,width=\dimmin{}{\dimwidth{0.90}}]{images/6}{}\mdline{189}%mdk

%mdk-data-line={192}
\begin{quote}%mdk

%mdk-data-line={193}
\begin{mdcenter}%mdk

%mdk-data-line={194}
\noindent\mdline{194}\textbf{Main Script.m}%mdk
%mdk
\end{mdcenter}%mdk
%mdk
\end{quote}%mdk

%mdk-data-line={196}
\begin{mdblock}{width=\dimwidth{0.50}}%mdk
\mdhr{}%mdk
\end{mdblock}%mdk
\begin{mdpre}%mdk
\noindent\%\\
\%~SeanChense\\
\%\\
clear~all;~close~all;~clc;\\
image~=~imread('cameraman.tif');\\
\\
[M,N]~=~size(image);\\
P~=~2*M;~Q~=~2*N;~~~~~~\%~zeros~padding\\
[v,u]~=~meshgrid(1:Q,1:P);~\\
u~=~u~-~floor(P/2);~\\
v~=~v~-~floor(Q/2);\\
\\
T~=~1.0;~~~~~~~~~~~~~\%~The~duration~of~the~exposure\\
a~=~0.025;~\\
b~=~0.025;\\
Duv~=~pi*(~a.*u~+~b.*v~+~eps);~\%~最小浮点数的精度,2.2204e-016\\
Hmov~=~T./Duv~.*~sin(Duv)~.*~exp(~-1j*Duv~);~\%~Complex value\\
\\
OTF~=~Hmov;~~~~~~~~~~\%~Select~the~OTF~(湍流或运动模糊)\\
subplot(231),imshow(log(1+abs(OTF)),[]),title('光学传递函数');\\
\\
FI~=~fftshift(fft2(image,P,Q));\\
blurFT~=~FI.*~OTF;~\%~Filtering\\
blurIp~=~real(ifft2(ifftshift(blurFT)));\\
blurI~=~~blurIp(1:M,1:N);~~~~~\%~Crop~the~border\\
subplot(232)\\
g~=~im2uint8(mat2gray(blurI));\\
imshow(g),title('退化图像');\\
\\
radius~=~500;~~~~~~~~~~\%~自行设定\\
\%~全逆滤波\\
fg~=~revfilter(g,OTF(1:M,~1:N),radius);\\
subplot(233)\\
imshow(fg,[]),title('全逆滤波');\\
\\
radius~=~0.001;\\
\%~逆滤波复原,半径(阈值)\\
fg~=~revfilter(g,OTF(1:M,~1:N),radius);\\
subplot(234)\\
imshow(fg,[]),title('半径限制复原');\\
\\
\\
\%~维纳滤波\\
wn~=~SCWinnerFilter(g,~OTF(1:M,~1:N),~0.01);\\
subplot(235)\\
imshow(fg,[]),title('维纳滤波');%mdk
\end{mdpre}
%mdk-data-line={245}
\begin{quote}%mdk

%mdk-data-line={246}
\begin{mdcenter}%mdk

%mdk-data-line={247}
\noindent\mdline{247}\textbf{revfilter.m}%mdk
%mdk
\end{mdcenter}%mdk
%mdk
\end{quote}%mdk

%mdk-data-line={249}
\begin{mdblock}{width=\dimwidth{0.50}}%mdk
\mdhr{}%mdk
\end{mdblock}%mdk
\begin{mdpre}%mdk
\noindent function~newim~=~revfilter(im,psf,radius)\\
\%==========================================================================\\
\%~逆滤波复原函数\\
\%~newim~=~revfilter(ima,psf,radius)\\
\%~ima:~原始图像\\
\%~PSF:~传递函数\\
\%~radius:~逆滤波半径\\
\%~newim:~~复原图像\\
\%==========================================================================\\
\%~若为rgb图像,灰度化\\
if~ndims(im)\textgreater{}=3\\
\preindent{4}im~=~rbg2gray(im);\\
end\\
\\
im~=~im2double(im);\%归一化/255\\
\\
\%~傅里叶变换(FFT)\\
Fim~=~fftshift(fft2(im));\\
P~=~Fim;\\
\\
[M,N]~=~size(P);\\
\%~逆滤波复原\\
if~radius~\textgreater{}~M/2~~~~~~~~~\%~半宽度阈值\\
\preindent{4}P~=~P./(psf~+~eps);~\%~全滤波\\
else\\
\preindent{4}\%~规定半径范围内进行滤波\\
\preindent{4}for~i~=~1:M\\
\preindent{8}for~j~=~1:N\\
\preindent{12}if~sqrt((i-M/2).\textasciicircum{}2+(j-N/2).\textasciicircum{}2)~\textless{}~radius~\\
\preindent{16}P(i,j)~=~P(i,j)./(psf(i,j)+eps);\\
\preindent{12}end\\
\preindent{8}end\\
\preindent{4}end\\
end\\
\%~IFFT\\
newim~=~real(ifft2(ifftshift(P)));\\
newim~=~(newim);\\
end%mdk
\end{mdpre}
%mdk-data-line={290}
\begin{quote}%mdk

%mdk-data-line={291}
\begin{mdcenter}%mdk

%mdk-data-line={292}
\noindent\mdline{292}\textbf{SCWinnerFilter.m}%mdk
%mdk
\end{mdcenter}%mdk
%mdk
\end{quote}%mdk

%mdk-data-line={294}
\begin{mdblock}{width=\dimwidth{0.50}}%mdk
\mdhr{}%mdk
\end{mdblock}%mdk
\begin{mdpre}%mdk
\noindent\%\\
\%~SeanChense\\
\%\\
function~newim~=~SCWinnerFilter(im,~psf,~gamma)\\
\\
im~=~im2double(im);\%归一化/255\\
\\
\%~傅里叶变换(FFT)\\
Fim~=~fftshift(fft2(im));\\
P~=~Fim;\\
[M,N]~=~size(P);\\
\\
for~i~=~1:M\\
\preindent{4}for~j~=~1:N~~~~~~~~\\
\preindent{8}P(i,j)~=~conj(psf(i,~j))./((psf(i,~j).\textasciicircum{}2)~+~psf(i,~j));\\
\preindent{4}end\\
end\\
\\
\%~IFFT\\
newim~=~real(ifft2(ifftshift(P)));\\
newim~=~im2uint8(newim);\\
end%mdk
\end{mdpre}
%mdk-data-line={335}
\begin{mdbmargintb}{4em}{}%mdk
\begin{mdflushright}%mdk
{\tiny\mdline{336}Created with~\href{https://www.madoko.net}{Madoko.net}.}%mdk
\end{mdflushright}%mdk
\end{mdbmargintb}%mdk%mdk


\end{document}
